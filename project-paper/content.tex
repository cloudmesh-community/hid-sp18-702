% status: 100
% chapter: Blockchain


\def\paperstatus{0} % a number from 0-100 indicating your status. 100
                % means completed
\def\paperchapter{TBD} % This section is typically a single keyword. from
                   % a small list. Consult with theinstructors about
                   % yours. They typically fill it out once your first
                   % text has been reviewed.
\def\hid{hid-sp18-702} % all hids of the authors of this
                                % paper. The paper must only be in one
                                % authors directory and all other
                                % authors contribute to it in that
                                % directory. That authors hid must be
                                % listed first
\def\volume{9} % the volume of the proceedings in which this paper is to
           % be included

\def\locator{\hid, Volume: \volume, Chapter: \paperchapter, Status: \paperstatus. \newline}

\title{Report: Big Blockchains}


\author{Lokesh Dubey}
\orcid{hid309}
\affiliation{%
     \institution{Indiana University}
     \streetaddress{3209 E 10th St}
     \city{Bloomington}
     \state{IN}
     \postcode{47408}
  \country{USA}}
\email{ldubey@indiana.edu}

% The default list of authors is too long for headers}
\renewcommand{\shortauthors}{L. Dubey}

\author{Gregor von Laszewski}
\affiliation{%
  \institution{Indiana University}
  \streetaddress{Smith Research Center}
  \city{Bloomington}
  \state{IN}
  \postcode{47408}
  \country{USA}}
\email{laszewski@gmail.com}


% The default list of authors is too long for headers}
\renewcommand{\shortauthors}{G. v. Laszewski}

\begin{abstract}

  Blockchain and specifically cryptocurrencies have been trending high
  and have been buzzwords in recent past. In particular, Blockchain
  and its popularity have been mostly tied with cryptocurrencies and
  all the related entities that comes into picture with any financial
  transaction, traditional or contemporary. Most of this attention
  remains channelized towards cryptocurrency and focuses primarily on
  the challenges of implementations and ways to overcome the
  limitations that may be subjected to cryptocurrency. In this study
  we do a detailed introspection on how Blockchain technologies are
  being used not just with cryptocurrencies but, also, how it is being
  applied into various other solutions where we treat Blockchain as
  open, distributed, peer to peer transacting, database. As,
  Blockchain is simply agnostic to any currency and can be applied to
  multiple other solutions which are not even closely related to a
  financial system. Further, we discuss that implementation of non
  financial systems can result into very large number of transactions
  and blocks to be created than it could ever be the case in any
  cryptocurrencies. And that would throw many more implementation and
  operational challenges at this new technology along with the very
  large amount of data being generated in form of Big Blockchain, Big
  Data in Blockchain.

\end{abstract}

\keywords{\locator\ Blockchain, Transaction Management,
  Decentralization, Middleman, Digital Autonomous Organizations}

\maketitle
\section{Current Economics and Transaction Management in Supply chain}

There have been extensive technological strides in most of what human
lifestyle interacts with on day to day basis. One can pick up any
product or entity around them and can always relate to the historical
advancements that have been done on it. It has been human nature
always to keep improving on the lifestyle and technological
advancements play a vital role in it. Most of these advancements in
one form of another have been about Internet of information. From
century old abacus, to an electronic calculator, to high performance
computers is just one example of it~\cite{tapscott1}. In everything
around us, one can find, that there have been smarter solutions
developed and the basic principle of any system has been altered to
make the devices smarter to make the efficient use of time and
resources. Most of the scenarios eventually find solutions in
information technology in unison with electronic advancements. And as
the name suggests information technology is all about information and
digitizing every single data that is around us and using it to its
fullest extent so that one can increase the efficiency of the process
and also, in the meanwhile, make the systems smarter by artificial
intelligence. However, all of these advancements, for instance, self
driving cars, smart grocery shops, or artificial intelligence, have
all been the result of Internet of information~\cite{tapscott1}.
Everything that is being done is mostly in form of making copies of
the physical information that we have at hand and making an impression
of it on internet. It has always been about digitizing the information
and passing it around.

One particular system, however, is extremely vital for the humans but
has not progressed a lot in principle~\cite{lionelshriver2}.
Transaction management and accounting of any form, even if digitized,
is still using the old methodologies and there is very little
advancements done on it is principle. For example. We can indeed pay
for our Uber (a Smartphone-Enabled Ride Hailing
  Service Alternative to Taxi Cabs) car ride directly to the driver
but eventually its all about transfer of a certain asset between two
parties~\cite{uber3}. In principle it indeed is just an alternative of
giving cash to the cab driver vs paying it via smart phone app. And
that includes a lot of intermediaries who are doing the job of
transferring the asset, in digitized form, to all the entities
involved and then depositing appropriate cash in the drivers bank
account. Even though it appears that this system works seamlessly
while enforcing a cashless economy if we look at it from a principle
that this is based upon, it remains the same. Eventually we are still
just transferring an asset, which is digitized, from one party to
another. Comparing this to the old system where you pay the cab driver
in cash one just has the comfort of using his phone to call a cab and
pay for it without worrying about cash. But eventually one is just
transferring cash. In fact to make this more beneficial one has
introduced a lot of intermediaries and entities in this complete
transaction which adds too much time for this process to be taken care
of~\cite{tapscott1}. When there are so many moving parts to a
transaction it will deter its performance. As an end result the Uber
driver who could have gotten the payment right away will get his payment
delayed or may be at the end of the month. Here are some use cases
that illustrate the problems with traditional transaction management
in case of supply chain and in the financial paradigm.

\subsection{Brexit}

Brexit is a term for the potential or hypothetical departure of
  the United Kingdom from the European Union. Brexit will result into
extremely complex trade predicament which would be put in place after
United Kingdom's exit from European Union~\cite{nicolasbotton4}. In a
very small amount of time borders and customs will have to be setup
between these geographies in a very short amount of time. The
traditional financial system, no matter, how much use of technology it
uses would either fail or would be extremely inefficient to provide
any timely transactions at such large level. Merely because of the
fact that an establishment which was so old and working needs to be
changed starting from a certain day where there are lot of
intermediate parties involved and many of them would be created viz.
regulatory bodies, logistical entities.United Kingdom's Brexit
team has however suggested a completely novice technology based
solution in form Blockchain to address this issue
\cite{nicolasbotton4}.

There have been multiple and mixed mentions of Blockchain in finding
resolutions to these complex problems in the UK. There are many
organizations which are encouraging the use of Blockchain and
Hyperledger technology a lot to find solutions to transform global
shipping sector. Hyperledger is an umbrella project of open source
blockchains and related tools. Biggest proponents of these
technologies have been IBM and Maersk~\cite{stratfor1}. IBM has
started a joint venture with Maersk and according to them they can get
the infrastructure ready if not complete but at least ready for this
transition in a period of six months. As Blockchain provides a safest
and quickest way to handle these supply chain models which keeping
track of commodities and assets without any misuse, with all the
possible right protections of tangible and intellectual goods these
technologies can prove to be really helpful.

There are however some difficulties and problems which have been
identified with applying Blockchain technology solution to these
issues as well. These flows which have to be applied for the custom
border transaction management and the smart contracts to be developed
for these technologies might be far difficult to implement than any
typical supply management problem. Even though the joint venture of
IBM and Maersk have committed a certain deadline to achieve this but
it does not seem like a reality. On top of the complexity of the
problem at hand the will to take these risks and investing
infrastructure and resources on these problems to solve something
which already has a tested solutions, but of course an inefficient
solution, would be difficult. It has to be British and Irish
governments both who have to agree on going ahead with this
implementation and have to take the risk of going ahead with this new
buzzword technology which may or may not work. It is simply not
possible to provide any kind of accountability for the solutions which
cab be applied using Blockchain because simply there is not much
information available to even visualize what is being tried. Secondly,
the technology itself is novice and has only been applied to simple
Hello world solutions and cryptocurrencies only like Bitcoin, Ethereum.
However, those transactions are fairly straightforward as there is
still a lot of skepticism about that as well. Which it the reason why
not a lot of people have been adopting these technologies.

Lot of governments have been in fact banning the circulation and usage
of cryptocurrencies because they are simply not ready for it
\cite{kate2}. There are multiple factors that apply to the readiness
however. Some can be attributed to the ignorance and typical lack of
knowledge of the governments, some are because they actually
understand that these currencies can take away a lot of control over
these pseudo currency on the internet which does have an affect on
real world.

\subsection{Information is duplicated, not commodity}

Internet of information (things), has revolutionized the impact of
technological advancements on human race. The impact of this
information on humanity can be imagined sheerly by the fact that now
oil is not considered as the most valuable resource, but, its the
data. However, most of this information is just replicas
\cite{economist5}. Any data we can think of that is on the internet is
a duplicate or a copy of something in real world. All the information
that we have on internet is either a blatant copy of something in real
world. For example, textual information extracted from physical
documents and is saved as files in internet, or a song which was sung
by an artist and is uploaded as an mp3 file. Or on the other hand it
could be the information that is derived from this real world
information and has not significance of its own. For example, Facebook
has around 2.2 Billion active users monthly. Who are sharing a lot of
information in form of photos which are theirs, conversing with
friends which are their real thoughts, and liking, disliking posts on
Facebook which is an information which is derived by their action on
this different information posts~\cite{statista6}. So as far as its
information it can be digitized, made multiple copies of, and can be
shared across internet. This, however, does not apply to commodity.


On the other hand there can be some commodities which can be
considered as commodity and are also digital. Some examples of such
commodities are intellectual property of any user. This can actually
be considered in various different perspectives. There can be some
organizations who are basically providing nothing but some tutorials
which are not really created in real world but are actually the write
ups coming from many experts or experience people who have actually
worked on those products and know what needs to be done with them.
Lets consider Quora (a question-and-answer site where
  questions are asked, answered, edited, and organized by its
  community of users), it is a social networking site however it has a
very different objective than any other major sites like Facebook,
Twitter. Quora is very similar to an old site Yahoo Answers but
that site is literally to get answers for any random question that one
can imaging and ask. Quora on the other hand is a site where a lot
people collaborate, socialize with people intellectually. In most
cases we consider it like an online book club but with millions of
users to it. There are many users on these site who write up really
nice articles on this site which are essentially a commodity. These
writeup are even published and cited by many publications, news
papers.

But in this case a well the matter which is being generated online is
completely open to all the users of quora because there is no content
on Quora that is hidden. Its just an intellectual property of a Quora
user and if the wishes to she should be able to charge royalty for any
usage of such property and most important should have control over any
illegitimate and abuse of the property by online users knowingly or
even unknowingly.

Unknowingly, because internet is nothing but the users essentially
jumping from one url to another without even knowing what is about to
appear on their screen~\cite{jane3}. The same hyperlink basically may
take you to a random site which is allowing a lot of ads to make
revenue. However those ads are nothing but url to some other sites
which most likely may be taking you to sites which are offering movies
and songs which are freely available or are simply pirated copies. In
these kinds of scenarios it makes it extremely difficult to figure out
who was actually consuming the content knowingly and was actually
looking for it. Or the person was simply driven into this mesh of
hyperlinks because of which the user is here. There are indeed a lot
of ways to tackle these problems but the most any government of the
copyright protection boards can do is shutdown the website. They
cannot really charge them or hold them accountable for it because they
were simply providing links to their ad providers and did not intend
to do that. These kind of loop holes in the existing system makes it
extremely difficult to keep the commodity, which indeed exists on the
internet in a virtual form, secure.

\section{Asset}

Commodity, on the other hand, cannot be really digitized and
eventually cannot be shared. For the sake of simplicity lets just
consider our commodity of concern for sometime as currency. This
particular theory, whereas, can be applied on any commodity as well.
So there have been a lot of technological advancements done in case of
banking in all this time as well and we have moved from the age old
barter system, to having a currency which basically means the
government of one's country owes the bearer of the currency the amount
of money mentioned on the currency. From direct handling and
exchanging of cash there were more advancements made in the case of
banking where the currency was indeed pseudo digital~\cite{Chen7}.
Yes, for a certain period of time it felt like the currency is digital
and you can exchange it and purchase real world things with it, but it
was not really digital. In this case as well, just like information,
the currency was indeed a replica of a real world currency sitting in
internet. On one hand indeed the transaction that was carried out was
digital but at the end of the someone else did the physical transfer
of money in any scenario. We can consider the same example as
described before where one can pay the uber driver with an app and do
not use cash at all. But indeed there is physical money that exists in
the real world which is exchanged at a different level, in this case
federal banks.

Now, lets take another look at this and flip the currency with a
commodity. Let us take an example of Youtube,  a free
  video sharing website that makes it easy to upload, watch online
  videos. Youtube creator space and advertisement model. In current model
there are lot of intermediaries involved in the complete payment model
\cite{bryanm8}. For example there are advertisers, youtube
organization, viewers, youtube creators, subtitle providers and many
more. Here the commodity is basically a youtube video in form of a
song or DIY (Do it yourself).\ which is uploaded on Youtube. Now
before the artists get paid there are a lot of unknowns and a lot of
hops that the money takes to eventually reach to the actual creator of
the content. Moreover with increasing competition from fellow youtube
content creators even though we have digitized everything and it has
been made easier for any youtube user to have very wide and quick
outreach to a lot of viewers in form of youtube platform but the
artists are getting a lot less paid than the traditional copyright
model~\cite{tapscott1,helienne9}.


As we can see in most of these scenarios there are a lot of flaws in
the system which makes it extremely inefficient. We can see that it is
indeed a natural progression of research to make life easier and make
the systems more efficient. But in most of these cases everything is
digitized from real word to information on internet. But in almost all
of the cases the underlining fundamental principle remain the same. We
do see some benefits of it because its easily accessible over internet
and has a much wider reach than it had every before but still there
are a lot of inefficiencies in the system which needs to be addressed.


\section{Centralized Intermediaries}

The biggest problem with the traditional transaction management
systems or financial institutions operating over internet are the
middlemen~\cite{torres10}. Here a middleman is some who exists in
practically every transaction that occurs in a supply chain model or
any financial transaction that happens in banking sector, stocks \&
bonds or it can simply be a tap a consumer does with his Visa card on
Walmart. In fact, the middleman existed right from the days of barter
system where if the commodity you need is with the person who is not
interested in your commodity can be solved simply by involving a
middleman who charges a certain amount and exchanges the commodities
for you and stores what is in excess with him to be utilized for a
similar transaction later. In today's world however this can be
considered as entities like Banks, National Security Agency (NSA),
Google which in most cases are centralized. As described before in our
previous example of the Uber cab driver payment we listed numerous
intermediaries down ranging from the app provider, banking
institutions, regulatory authorities. Let us take one more example
of remittance of currency to another country. In today's highly
developed and technologically advance system as well it takes at least
5-6 approximately to send money from one country to another
\cite{martinez11}. There are many intermediaries in this scenarios
like the sender, sender's bank, sender's remittance bank, federal
bank, authority from sender's country, federal authority from
receiver's country, receiver's remittance bank. And even though highly
digitize these all intermediaries are still a bottleneck in making
these transactions efficient and faster, because of the sheer fact
that they bring any transaction to a centralized ecosystem and it can
be validated, rejected, updated only from that centralized system.

Blockchain as a concept and the related technologies are primarily
taking out the middleman out of the picture, inherently~\cite{lloyd4}.
As the blockchain concept in itself is based on the fact that its peer
to peer transaction, which can be regulated by regulatory authorities
but it would not be governed, approved or moved by the middleman. As
blockchain technologies and the principle behind it in form of a
distributed ledger basically takes away any kind of need of a central
authority or an approver to complete the transaction. One of the
postulates of these kinds of establishments later would be that
because this is a peer to peer transaction the participants can decide
the amount they need to charge or can actually get to agreement
faster, with mutual consent, and most importantly whenever they want
to.

Middleman in most cases is not really the person who is even in the
transaction. A transaction, if we juxtapose it with Barter system, is
more or less necessary just to involve two or three parties who really
wish to exchange a particular commodity with another. One commodity in
most of the cases would be financial asset like currency. But in most
case there are two person involved in any transaction where one is the
provider and has either excess of a certain commodity, or grows that
commodity and makes a living out of it by selling it, or is planning
to get rid of that commodity for some reason. And the other person
involved in these transactions is the one who is actually in need of
this commodity. But where does the middleman come into this picture.

Here middleman is basically a finder, minder and grinder. There are
many issues when these commodities are to be exchanged. First of all
there has to be a way to connect these two participants who are trying
to exchange commodity but there is no mutual channel between them so
that they can share that information. Second the commodity which is
being sold may too much or too less for the other participant who
needs it. In whcih this same transactions becomes a little more
complicated with two participants, who are to be found first, will
provide the commodity to just one participant. This is a fairly simple
and very basic problem statement in that a middleman helps a lot by
taking the risk and finding the participants for a transaction, making
sure the appropriate security is maintained and both participants are
feeling positive about transactions and are rest assured of any
fraudulent transaction.

In any of these scenarios we can find that any transaction related to
any commodity when involves a lot of middlemen the system becomes
inefficient irrespective of how digital or best the solution is
because at the end of the day the commodity has to exchange this
multiple hands, even though digitized. And this takes a lot of time
and resources. What this showcases is that even though most of the
information and commodity digitization has improved the speed of these
transactions to a certain extent but in principle they are all
following the same traditional real world methodologies and are
centralized bottlenecks.

A research paper that was published in 2008 focusses only on one
certain aspect of currency called Bitcoin~\cite{bitcoin12}. However,
Bitcoin is merely one form of commodity for which there is a Blockchain
which is active and people own and transaction in that Blockchain to
carry out transaction of their Bitcoins. The principle on which
Bitcoin is based is something which is extremely novice and provides
solution to all the limitations of the traditional transaction
management systems mentioned above and is called Blockchain. Which
provides a way to devise decentralized and distributed database that
allows peer to peer transactions by removing the middleman, is
agnostic to any international boundaries and can be applied to any
kind of commodity, asset, intellectual property and not just money.


\section{Blockchain}

Blockchain is a distributed database in form of a ledger which is
updated with distributed transactions that are secured by cryptography
and works based on the consensus of all the parties involved in it
\cite{beck13}. A Blockchain is simply a immutable ledger of various
transactions that are stored in that ledger and any transaction can
only be added to the ledger after its validated by all the parties
involved and is secured by cryptography to avoid any foul play. These
ledger in itself consists of various blocks which are a set of
transactions which are combined together in form a Block and are
appended in an already existing chain of Blocks. To process these
transactions and to add any block to existing block chain there is a
certain amount of computational work has to be put in to figure out a
certain cryptographic key which can validate that next block is the
correct block to be appended to the chain. Blockchains are generally
open to all on the channel to see and process transactions for all
users known as miners.

Rather than delving into the details of what Blockchain is and to
remain consistent with scope of this study the focus would here after
would be more on the benefits of Blockchain, their applications and
how they open new avenues for Big data. At every transformation that
Blockchain can bring we will look at each aspect it from both
perspectives, monetary and non-monetary both. As, this has been the
premise of this study right from start that Blockchain is not just
about cryptocurrency or something monetary but it can be applied to
any commodity, use case where there are transactions~\cite{beck13}.

In this further exploration we will look in various aspects and issues
in real world that blockchain technology has more or less uncovered.
Some of the major industries and domains which this technology can
really help with our the global financial establishment, a genuine
true global world economy, while making sure that the security is not
compromised. And providing appropriates hooks for all kinds of
security auditing so that there is always no question or hesitation in
any adopters to be skeptical about the technology and to only focus
and invest energy in getting the best out of this technology.

This particular study, right from the start, have been insinuating a
lot on not considering Blockchain as a cryptocurrency driven
technology. In fact there is a high possibility that Bitcoin like
cryptocurrencies which are highly driven by the market are extremely
sensitive commodity to be handled by blockchain~\cite{iyke5}. However,
the sheer concept of distributed database which immutable in itself
makes a very good case of, if not cryptocurrencies, blockchain
technologies in themselves have a very good future. Many organizations
and big supply chain management companies have already started
garnering benefits of the available solutions around blockchain.
However, there are not at all chances that cryptocurrencies will not
survive. Its just the general human tendency to be skeptical about
anything related to finance. But still cryptocurrencies have a very
bright future as they are and will complement any service that is not
centralized and not to rely on any legacy on payment systems.

In the interest of keeping this study scoped and because there is
already a lot of work that is mostly done on Bitcoin and
cryptocurrencies in the subsequent topics as well we will focus mostly
on the real world, non cryptocurrency applications, limitations,
challenges of Blockchain~\cite{yli18}.


\section{Global economy and prosperity}

As we have established that the Banking sector has been digitized at a
very aggressive rate nearly everything in a bank these days is either
available through internet and provide easy access to multiple
services of a bank over a computer or a mobile phone. As glittery and
ambitious it may sound the underlining systems are still following the
same old traditional approach to handle any kinds of transactions. And
anything in banking security eventually can be related to the identity
of the transactions. A bank will require a lot of identifications
before anyone can get a bank account~\cite{cocco14}. This is one of
the major flaws in the current banking and financial systems. Anyone
who wishes to be a part of the global economy should have an identity.
Even though the lender and the receiver both have no business
whatsoever with providing their identities, their address proofs or
even their names to the bank. For a simple transaction like
transferring money from one bank account to another, the intermediary
here, the bank demands both of the parties to identify themselves.
Which simply gets extremely complex if one wishes to work on an asset
transaction beyond the geographical boundaries of a country or a
union.

This is simply not possible and it does not provides a growth-centric
environment which facilitates and promotes prosperity in the world.
There are still a large population in this world which does not even
have a bank account. More than 15 countries have only 15\% or less
people have a bank account. This does not mean that this population
does not require money or is simply living off a barter system. Vast
majority in this remaining 85\% of population simply cannot afford to
have identification because of procedural hassles, corruption in the
countries or they simply think its not of a benefit to them
\cite{camilla15}. In addition to this there are majorities who are
simply do not trust banking systems with their less but hard earned
money. However, in most scenarios while in a transaction the identity
should not matter just like an economy with cash. When you have cash
and if you want to exchange it for some commodity the seller does not
demand an identification. Of course there are various caveats to this
theory but for those kinds of problem statements as well controlled
audits can be arranged to avoid any misuse.

In contrast, if we take a commodity in consideration after all the
technological advancements and global push towards building a
prosperous and better world, we still see famines and malnutrition
being an epidemic in most of the under developed countries. Even
though there are organizations like United Nations,
Unicef which after being non profit organizations
dedicated for the betterment and uplifting of world's economy there
are still countries where large set of population dies of hunger and
does not have reliable source of clean water, food and clothes
\cite{bbc16}. This is primarily because of a broken supply chain
system which has a lot of intermediaries, and however with a good
intention, when this model constitutes of many countries it makes it
extremely complex, inefficient which causes delays~\cite{van16}.


Blockchain can provide various ingenious solutions to these problems.
On banking perspective in blockchain a simple peer to peer economy can
be established where an identity is only exposed to an extent that its
needed~\cite{yli18}. One does not need to have a government authorized
document to identify herself. Anyone with a small device that connects
to the blockchain can carry out transactions directly, in real time,
even with least money and denominations possible. With Blockchain a
free and open economy can be raised and established where people who
never had a bank account can make transactions across the world
without ever having to even show their identification. All that
identifies them is their private key for their account.

In juxtaposition any supply chain model for any of the relief that is
being sent from across the world can be driven and governed by a
central blockchain channel where smart contract govern what relief was
committed, sent and keep track of its location in real time
\cite{christ19}. Multiple auditors on the channel can ensure the
timely delivery of the good and services which can make the system
void if any of the intermediaries are causing delays in which case the
transactions can be voided at anytime. This can avoid delays
completely and drives the complete system based on performance and
inherent accountability where one only gets paid when work is done in
a timely and efficient manner.

\section{Shades of Identities}

Identity theft and misuse of personally identifiable information has
been one of the biggest threats in recent past. With the emerging
social media and Internet of Things it is next to impossible to be
discrete about information being shared on the internet
\cite{reuters20}. It is not only just about the social media but there
are millions of apps which are on one hand very useful and saves a lot
of resources and energy but on the other hand makes every user
extremely vulnerable towards their usage~\cite{statista21}. Internet
however has become one of the most easiest target for stealing
personally identifiable information and a lot of information which can
prove to be costly to the users~\cite{hedayati22}.

Most of these identities are already pretty much exposed in the
internet world and are being misused all the time. These identities
are used and misused as seen fit by multiple users of the internet.
Even for the sites like Facebook and Twitter which are considered
social networking sites the data is always at risk. One would
knowingly put all his personal information on his social media account
to share with his friends and family but still that information is
uploaded on the internet and has no control over its security and it
can be in fact misused and have been misused very easily. On one hand
with GDPR (General Data Protection Regulation)\cite{gdpr10} European 
unions are trying to get rid of any PII\cite{pii9} (Personally 
Identifiable Information) information to get out on any other country's
infrastructure even on cloud but on the other hand with these social
medias and mass utilized sites on the internet there is no control over
this information being shared with others.


There Is indeed a fundamental flaw in the systems that we use today
where we have to to share our actual identity in most scenarios to be
a part of any product that is on internet. For example in financial
and banking institutions we should be able to only share the
information that pertains to and is needed by the banking firm. It
does not have to be completely information where you share your full
name, your email address, your address and other details. Similarly if
someone is utilizing let us say cable from the cable provider. In which
case the person does not have to share anything else with the cable 
provider other than the address and may be a way to identify himself.
A very novice way to handle this completely by Blockchain is 
that it provides control to each user to have 
complete discretion over her identity and
the user can choose and decide to divest that information on that
particular channel or not, or in other words is that information
really needed to be shared~\cite{alex23}. In these implementations one
can really own and control the identity, trust the system not the
other user on channel as the trust arrives inherently by mutual
approvals.


\section{Ownership and Royalty}

With the explosion of information and rise of social media there have
been a complete paradigm shift in ways the different businesses and
domains play out. Social media explosion and information sharing
capabilities have provided a wide outreach for very small creators who
could never get to share their skills or benefit from them because of
the competition and limited resources. With platforms like
Twitter, Facbook, Instagram, Youtube now small creators
and artists whose work could never see the day light and garner
appreciation merely because of limited outreach, are provided very
large number of audience from their homes and hand held devices
\cite{saleem24}. However, all of this comes with a price. For example,
a pottery artist who could simply create his art and sell it locally
in a shop would earn much more profit than he could ever do that on
internet or Instagram by sharing his art to millions of followers and
devise a merchandize model on top of it to sell the product out. It is
understood that it is indeed difficult for anyone to sustain a business
in real world in a shop vs no cost internet but the comparable earning
of the creator is too low simply because of all the middlemen.
Instagram, Youtube themselves have their revenue models, to devise a
merchandize model one will have to hire more external vendors who can
operate on the sites incurs a lot of cost and the artist gets his
final share after paying off all the middlemen. Let alone the time it
takes for the final price to arrive in creators bank account which
takes a very long time~\cite{margaret25}.

With Blockchain implementation in these different domains a straight
forward peer to peer to transaction mechanism which gets settled in
real time. For example a song uploaded on Spotify, Youtube,
Soundcloud will basically be tied down on the \% composition of the
revenue model and the artist gets paid in real time based on the
number of views or listens on spotify. And it is the artist who decides
the price of single online playback of the song, or the usage of the
song in some other productions which may require royalty to be paid to
the artists. With blockchain implementation all of these transactions
can be settled in real time with the help of smart contracts and trust
less, handshaking transaction mechanism.


\section{Protection of rights and intellectual property}

After discussing identity theft and solutions which Blockchain
provides there are some other concerns in security as well on the
information sharing platforms of social media. As very very large
amount of information is being uploaded to the internet, where around
8 exabyte of data is being created only on phones~\cite{jeff26}. On
the internet virtually there is no restriction on the copyrights and if
some one wishes to upload their copyrighted content or intellectual
property. Any content that is uploaded on the internet can easily be
duplicated and pirated with an extreme ease even today~\cite{anita27}.
There are technologies working really hard to make this happen but it
has always been a back and forth between tightening the security with
various technical advancements and the internet users looking to hack
the systems, finding loopholes and exploiting them are gaining equal
amount of expertise and technical help. There has never been a robust
way to protect the rights on any intellectual property or content on
the internet. It can be the movies, songs, copyrighted documents and
much more.

The biggest hurdle in enforcing this is in fact the geographical
limitations of jurisdiction across countries which either does not
exist or takes a lot diplomatic deliberations to achieve
\cite{anita27}. And that consumes a lot of time to enforce anything
which in the this age of exponentially growing internet which is
bolstered by Cloud and as a service products becomes mute. The as a
Service concept, where companies offer services to help other
companies  become more efficient, offers a path to reduced costs and
streamlined workflows. For example, the sites which are pirating the
data and are distributing freely over internet just to get more
advertisers never really get blocked because of similar
limitations~\cite{ian28}. Even if they are identified and shutdown
they can easily relocate themselves to any other data center of any
Infra/Platform as a Service provider. 

Blockchain can prove to be the best solution in this case where any
intellectual property is considered an asset and is treated as an
asset where any user of the internet trying to access this asset will
have to authorize themselves and pay, in case it requires payment,
before they can even see the data~\cite{bitcoinist29}. As the data
security would not really be driven by the application that is hosting
the data, in fact the applications can be created without any
authentication mechanism in the first place and an asset based
authentication and authorized can be applied using Blockchain
channels.


\section{Halcyon ways of business}

Most important aspect that traditionally business minds use to follow
was to keep everything local. Old business model which many big
businesses followed was to reduce cost of raw material and cost of
transportation of raw material, build everything locally. For example
a car manufacturer would like to produce the raw goods required to
make the car locally to avoid transportation and higher margins
charged by other providers. This however have changed recently.
Companies now, considering the high demand and supply, try to pay
another companies to manufacture the complete goods and just ship them
as their labels~\cite{ewan30}. However, in these business models as
well there are a lot issues where a suppliers reliability on
availability, sustenance of quality and cost can still be an issue.
The major issues in these cases are again the intermediaries. Because
a provider from whom a business is getting their finished could very
well be following the same business model and is primarily dependent
on another provider. This chain can be really hurtful for any business
where a single point of failure can cause huge delays and hurt he
reputation in the market.

Blockchain solutions can provide the goodness of the both the old
solutions and the contemporary ones. It is indeed not a good idea to
start manufacturing bricks as well when an eventual goal is to build
houses. However blockchain solutions can provide a new twist to the
chain of providers which are now bound by the smart contracts of
blockchain channels and live biddings can be associated with contracts
to avoid any delays from any of these providers.

\section{Disaster Recovery an overkill}

This is a staple benefit of anything decentralized but let us consider
this particular scenario in detail. There have been a lot of companies
which are running major platforms which provide disaster recovery
tooling for mission critical infrastructures which cannot afford to
lose any of their data, intellectual property or even the service
availability. These organizations can vary from being financial
institutions which are holding records of and money itself for
billions of people and are handling all the transactions, there are
major services which are running cloud services on which services
there many applications which are running their services as a
platform. However, one of the biggest problem and fear of these
systems is that if they are static and in house it is a responsibility
of the organization itself to maintain and implement disaster recovery
processes for their infrastructure. If its basically on Cloud then it
all depends on the DOUs and the terms and conditions that one agrees
to before going on cloud so there is no sense of security there. Cloud
can provide easy backup and recovery processes because of
virtualization. But the kind of catastrophe we are discussing here is
worst case scenarios, natural calamities where everything in that
town, city, state of a country is lost.

There have many natural calamities in which complete infrastructure of
towns, and multiple cities is completely lost to the nature and its
extremely difficult and tiring of some countries to even get the basic
amenities like power, water supply running, let alone someone
looking into recovering he the data of a data center. There are,
however, many disaster recovery solutions available which are
providing these services to replicate the data from one data center
remotely to a totally different data center. Replicating the data
locally enough with multiple exclusive pods so that they are running
on completely separate infrastructure. In these kinds of scenarios the
disaster recovery services charge hefty amount of money as they would like
to invest more and more based on what category of recovery one would
want. One being where a simple recovery from a lost hard drive is
needed and highest level of recovery where if the complete site or
datacenter is lost one should be able to recovery everything from a
different site.

Blockchain technology can provide a centralize database which is
inherently in principle distributed. In fact these are the kind of
database which are replicated across the world in there default
implementation itself and do not really have to be thought out
specifically for disaster recovery. When there is any disaster the
continuity of the business stays as is because even though some of the
nodes on the channel are lost but there are still always the rest of
the nodes which have complete data on them. It provides a
$24 \times 7$ support through out the year without applying any
complex technologies like replicating the data on a different
continent or in a different city. Also there are smart contracts which
are capable of storing the transaction related code and will trigger
on provided conditions to alert, roll back transactions, procedures,
network lockdown and backup procedures~\cite{elmer7}.

Disasters can be of some other forms as well like if there are
malicious attacks done to the internet facing infrastructure and if
there are Denial of Service (DOS)\cite{paloalto8} attacks which 
can render the targeted systems down.
In these scenarios as well the distributed
setup of blockchain channels can really help keeping the system up and
running. Even though there is a long way before blockchain technologies
can be polished more for these kinds of totally unrelated looking
applications but with a very high interest of major technology
entrepreneurs and businesses all that has been shared as a proof of
concept and research articles it would not be long before which
blockchain can be used as reliable, secure, highly available and
peer-peer fault tolerant infrastructure.


\section{Big Blockchain}

We have explored numerous possibilities that can overcome the
limitation of todays information sharing world and how blockchain
technologies can be applied to these contemporary solutions which
showcases a futuristic path towards a truly shared economy over
internet, built on mutual trust, ensured by cryptography while
protecting the rights to intellectual properties and assets of
billions of the users all over the world. However, as it appears to be
the case, introduction of Blockchain technology to solve all these
problems also poses new challenges on the adopters of these new
technologies which we have to be prepared for~\cite{smith31}. These
challenges do not pertain to the challenges of enabling more features
and capabilities of blockchain. But on the other hand these are the
challenges which will have to be addressed before the blockchains got
beyond our control. Let us introspect some of these challenges, the
causality and status of solutions to address these challenges, if any.

\section{Auditing challenges in Big blockchains}

On one hand it can be safely said that Blockchain technology's
substantial benefit is decentralization. In spite of that it opens
numerous avenues of improvement on the current world economics and
supply chain model which poses their own challenges which needs to be
well thought of before hand. Protecting rights of a user's
intellectual property on a blockchain based channel seems like a quick
and viable solution but if we increase the cardinality of this system
to billions of users on the internet it can very swiftly turn into a
system which, even though secure, would still require a lot of
auditing and safety fall back mechanisms~\cite{smith31}. Which are
devised in a way that the issues are quickly and easily identified
before they can go out of hand. For instance it is simple enough to
say that a particular song was uploaded on the Spotify, Youtube,
Soundcloud by a creator and it was made available to be listened to,
downloaded by all the users on the internet in a peer to peer
transaction mechanism. However, the sheer amount of data that can be
generated just for this one kind of transaction could be huge and the
length of the chain may start to grow very aggressively. Having a
control over that much amount of data, for appropriate auditing, to
make sure the creator is getting paid her fair share in real time,
there are no delays in processing of the transaction appropriate
BigData solutions have to applied on Blockchains too.

Blockchain as distributed it may be and the mutual trust it is based on
and is backed by cryptography, at the end of the day its a chain. And
to follow any kind of trail for any asset or commodity, or users for
unforeseen reasons might not be simple enough once the adoption of
blockchain has matured enough to be an integral part of the world's
economy and financial institution. From protecting rights, to
protecting any illegitimate use or wrong doings, to preventing delays
based on smart contracts. All has to be settled and has to be
auditable at one point in time. In all fairness the transactions on
the blockchain will eventually have to be corrected, and would require
transaction malleability~\cite{daniel32}. Digital Autonomous
Organizations would come into existence and would pose
another challenge for auditing as these organizations do not really
come under any jurisdiction or law. Simlar, challenges can also be
posed by Long-term and short term blockchain forks where the chain has
been forked into multiple chains.

On one hand it is being considered as one of the salient points of
blockchain where it does not come under any specific countries
jurisdiction and can seamlessly integrate and allow peer-peer
transactions over internet across the world. But these solutions do
pose audit challenges for a particular country as it cannot go beyond
its geographic limits to enforce its policies. Same situation would be
exacerbated with cross chain transactions where there are transactions
which are carried out between two separate self sustained chains.

\section{Challenges with Blockchain and Cryptography}

There are already a numerous challenges which are posed against this
fledgling technology at a very preliminary level. After adoption of
blockchains, as discussed above, at a very large scale and creating an
impression in almost all of the global economy may pose its own
challenges beyond auditing. There are some basic challenges even today
around the technology readiness of our industry to be able sustain
this mammoth task. On the oher hand the basic principle behind
blockchain for investing compute and providing power to computer is
raising a lot of red flags already from the conversationalists. Also,
beyond auditing, governments of any nations would not easily digest
this fact there are pseudo economies, businesses running under Digital
autonomous organizations which has no real world presence and no one
in real world can be held accountable for the misdoings of such rogue
organizations~\cite{myungsan33}.

On the other hand there is a an issue pertains and affects the
development and enthusiasm towards blockchain i.e.\ losing jobs to
blockchain~\cite{michael34}. Not very different from the scenario when
computers arrive, there was harsh criticism towards computers too from
a range of communities concerned with people losing jobs to computer.
Security as well has been one of the major concerns in the blockchain
paradigm~\cite{mauro35}. However, when the idea of blockchain was
published and the way it was devised it is nearly impossible for
someone to breach any of the block chain and make changes to it which
can sustain~\cite{bitcoin12}. As the whole model is based completely
on the cryptography and it requires a certain amount of compute spent
on it, it is nearly impossible for someone to process a fake
transaction and add the block to the chain. Because even though the
block might get added it may never get trusted by all the other
channel users. Basically the malicious user trying to hack the system
will have to put in enough compute so that he can race against all the
compute that is being used by all the miners in the world for that
particular blockchain. And not just that to make the chain bigger
enough to trust the same user will have to keep increasing his same
chain with multiple block addition before anyone else in the channel
does it.


Whilst we explored the nearly endless possibilities in which
blockchain technologies can be leveraged and can prove to be able to
provide smart solutions and applications to a wide spectrum of
information technology world. There are still a lot of challenges
ahead for this fledgling with a bright future technology. Ranging from
the fact that there is a still a lot volatility in acceptance of the
technology by all just like it was with Cloud computing when it came
to energy consumed to do the computation work to make the work be
validated. There are many challenges which various technology houses
who are endlessly focused on building smarter solution to the nearly
gone out of hand internet of things and make the world economy for
global, secure and controllable.

\subsection{Technical advancements}

Technology not being ready is one of the major concerns in most of the
cases. There Is a lot of enthusiasm towards blockchain and the
technology powerhouses, businesses, entrepreneurs are continuously
working on expanding the capabilities of blockchain but with the tech
world in a state where right now everyone understands that it has
bright future but it will still take sometime to be mature enough to
be widely acceptable. Primary reason for this is that there are a lot
of efforts being made in this direction but the focus has been fairly
distracted. Ranging from getting the cost lower to implement it,
making it truly decentralized, and to make really energy efficient.
There have not been a single chain yet that implements all of these
together. Any chain that is fast and is truly decentralized would
require a really high cost of implementation because of the
appropriate infrastructure needs.

\subsection{Energy consumption}

With blockchain technology and the underlining principle of
cryptography and putting in cpu work to validate a certain block on
the block, at least in these scenarios is considered a waste of power.
The world is already having discussions on various sources of energy
that while being produced and even consumed are causing global warming
and climate change on earth. In that context introduction something
like blockchain which needs a lot of compute essentially to make the
channel secure is a difficult imposition to be widely accepted. But
there are a lot of initiatives that are being carried out right now
from companies like IBM, Intel which are continuously working building
green blockchain models and implementations which are energy
efficient.

\subsection{Governments}

There are many governments of various nations who see a lot of
benefits in the emerging technologies and Government of the USA, China
are trying to adopt the technology and are trying to get on board
while its still growing so that the infrastructure needs are satisfied
in an early stage and they are prepared for it. The primary focus of
this study has been mostly on non cryptocurrency application of
blockchain and the government views on that aspect of it has been
fairly positive. It is the cryptocurrencies and the self sustainable
pseudo currency which can be exchanged across the boundaries of the
countries has been one of the biggest areas where government may want
stifle excessive growth in that direction.

\subsection{Jobs}

Blockchain technology's underlining fundamental is to remove the
bottlenecks and make the communication peer to peer. Which highly
drives in a direction to remove the middlemen and all the
intermediaries involved to an automated balanced systems of Smart
Contracts, Transactions to do the auditing, approvals and any additional
processes. One of the challenges which this, and most of
it is due to lack of education as
well, is that adoption of blockchain would cause a lot of job loss.
This is not a new aspect of growth in the information technology
industry. When the internet was booming and even before that when
computers where getting more and more popular in the industry almost
all were worried about the job loss by automation. But eventually
there is a broader answer to this problem and it is that yes there
might some jobs which will reduced to a smart contract code which
does, let us say an approval, but in most likelihood this is only going
to be shift of job to something of more importance while building the
planet smarter with a truly shared economy. However, it has to be
understood that to keep up with technological advancements everyone
has to keep themselves updated so that the day something that one does
is automated new avenues will open and one should remain updated and
ready to capitalize on such opportunities. There have been various
studies being done to figure out the impact of this new technology on
the jobs but it has mostly been superficial and qualitative, not
quantitative~\cite{michael34}.

\subsection{Too autonomous}

One of the other challenges of blockchain technology is pertaining to
Digital Autonomous Organizations and they going rogue and building
their old self sustaining economy once the marriage of artificial
intelligence happens with blockchain. There have always been a lot of
discussion around automation, robotics and artificial intelligence
taking over the world. Which has mostly been a discussion and there
has been no legitimate study on how much truth it is to it. It is
indeed not possible to predict something like that because that
implies that we can predict how our technology is shaping up. It is
indeed true that technology is mostly driven with a common cause and
intent of the drivers of the technological world like technical
institutions, research institutions, businesses. And it highly depends
on their interests and a balance between their intent behind shaping
some technology. It could be for the common good of humanity but the
same can apply to someone with a malicious intent too.


However, every now and then after all the security measures, whenever
there is any cryptographic attack which goes viral the prices of
cryptocurrencies suddenly start crashing. There Is a certain truth to
it but indeed in principle the technology is safe~\cite{david36}.

\section{A truly shared economy across the world}

Blockchain development have been always driven with a good intent to
overcome the shortcomings of the world's current status in form of
misuse and distrust because of aggressive growth that drives a lot of
collaborative with multiple parties involved which ends up being a
bottleneck and does not provide the right value of the asset to the
producer but in fact gets a lot of intermediaries who make the system
complex and stifle the growth from the end producer and consumer
perspective. With blockchain it would be possible to get the whole
world on a single shared economy not only from the monetary aspect of
it but from a supply chain perspective as well where the geographical
borders and authorities do not drive the common intent of the world
towards progress. Here are some insights on how the world might look
after it.

Blockchain like technology and continuous endorsements and
advancements in it will cut down the monopoly of certain big
organizations who are now part of this recursive loop where they keep
pushing their agendas and ideas with their financial capital over the
industry. Blockchain promises that it will change this completely and
will render the intermediaries, middlemen and all those bottlenecks to
simple auditing, smart contract automation. The technology will
provide a women with no access or enough money to own a bank account
or even a national identification, her own place in shared economy
where she can easily transact with a person across the world without
any long stretching processes. It will be able to provide that
musician who has been uploading music to the internet but does not
really get the right value of his product because of unauthorized use
of his intellectual property. The musician will be able to transact
directly with each listener and charge accordingly rather than a
generic rate applied and governed by sites like Youtube, Spotify.

And most importantly this will provide a shared economy across the
globe in which without having a middleman any small or large business
will get the benefit of purchasing the raw material directly from the
producer and saving a lot of resources which make this complete
transaction as profitable as it can be. Which in turn boosts a
collective business revenue.


\section{Conclusion}

Keeping the non-cryptocurrency applications of Blockchain technology
in this study we found that there is a huge endorsement from the
technological entrepreneurs, businesses and governments on adopting
and to continue advancement of blockchain technologies. It was
identified that the intent and underlining principle of Blockchain to
eliminate bottlenecks, intermediaries, middlemen is mutually shared by
a numerous of organizations. We identified that the technology is
moving towards highly ambitious goal of transforming the global
economy to a shared and prosperous economy, protecting the rights of
the intellectual property, making sure that the creators get their
right compensation, to own and monetize data. We discussed a few
heterogenous applications of blockchain which showcase how beneficial
and generally application this technology is. We discussed in detail
how the middlemen are hurting the collective growth of economy and how
blockchain and its related solutions and advancements can help with
it. For example, how blockchain can be really helpful in a quick
solution to the Brexit problem of customs transaction management with
a new geographical border being assumed between United Kingdom and
Ireland. We also discussed with numerous examples how blockchain
solutions can help with securing identities, ownership of creator on
his intellectual property, and in providing the right value of
creators product. We also delved in detail on how against the common
perception its not just the blockchain application to BigData world
that is important. But for technological advancements of Blockchain
and for it to be able to sustain the kind of applications that are
being built based on Blockchain as an underlining solution may require
a lot of progress in Big data in Blockchain as well to be able sustain
the aggressive growth of Blockchain in future without compromising
with any of the underlining principles.


\begin{acks}

The author would like to thank Dr.~Gregor~von~Laszewski for his
support and suggestions to write this extended report.

\end{acks}

\appendix

\section{Appendix Descriptions and Acronyms}
\begin{description}
\item[Unicef.] The United Nations International Children's Fund is a
  United Nations program headquartered in New York City that provides
  humanitarian and developmental assistance to children and mothers in
  developing countries.

\item[NSA.] The National Security Agency is a national-level
  intelligence agency of the United States Department of Defense,
  under the authority of the Director of National Intelligence.

\item[Twitter.] Twitter is an online news and social networking
  service on which users post and interact with messages known as
  tweets.

\item[Facebook.] Facebook is a social networking site that
  makes it easy for you to connect and share with your family and
  friends online.

\item[Instagram.] Instagram is a photo and
  video-sharing social networking service.

\item[Spotify.] Spotify is
  a music, podcast, and video streaming service.

\item[SoundCloud] SoundCloud is an online audio distribution
  platform.


\item [Digital Autonomous Organizations.] Digital Autonomous
  Organizations is a decentralized autonomous organization,
  sometimes labeled a decentralized autonomous corporation, is an
  organization that is run through rules encoded as computer programs
  called smart contracts.
\end{description}

\bibliographystyle{ACM-Reference-Format}
\bibliography{report}


